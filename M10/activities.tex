\subsection{Activities}

%%%%%%%%%%%%%%%%%%%%%%%%%%%%%%%%%%%%%%%%%%%%%%%%%%%%%%%%%%%%%%%%%%%%%%

\subsubsection{GPU shaders}
\begin{itemize}
	\item Decoding shader was written.
	\begin{itemize}
		\item This is to visually see the encoded data.
		\item Encoded data does not make a lot of sense visually.
	\end{itemize}
	\item Disparity map shader was written.
	\begin{itemize}
		\item Results are not as good as they can be.
		\item A different approach is being written.
		\begin{itemize}
			\item Current approach allows to check only for 11 levels of disparities.
			\item A more flexible and complex shader/pipeline is being written.
			\item It is needed specially because the distance between cameras has not been decided yet.
		\end{itemize}
	\end{itemize}
\end{itemize}

%%%%%%%%%%%%%%%%%%%%%%%%%%%%%%%%%%%%%%%%%%%%%%%%%%%%%%%%%%%%%%%%%%%%%%

\subsubsection{Integration of both cameras}
\begin{itemize}
	\item \textbf{Rewired.}
	\begin{itemize}
		\item Both cameras are clocked from the board with the STM32F429ZIT6.
		\item Both cameras' PWDN pins are connected to the same STM32F429ZIT6 pin (previously connected to ground).
		\item Remaining pins were connected to their respective microcontroller's board.
	\end{itemize}
	\item \textbf{Sofware related-work.}
	\begin{itemize}
		\item Synchronization logic was added.
		\begin{itemize}
			\item On both ends (STM32 and Raspberry).
		\end{itemize}
		\item It is very important to get both images at the same time.
		\item PWDN pin is used to:
		\begin{itemize}
			\item Halt internal device clock.
			\item Reset all internal counters.
			\item Keep registers.
		\end{itemize}
	\end{itemize}
\end{itemize}

%%%%%%%%%%%%%%%%%%%%%%%%%%%%%%%%%%%%%%%%%%%%%%%%%%%%%%%%%%%%%%%%%%%%%%

\subsubsection{Port to STM32F407VGT6's board.}
\begin{itemize}
	\item I am temporally going to work on this board too and the code had to be ported.
	\item SPI, DCMI, DMA and EXTI are now working on the other board too.
	\item Every pin had to be remapped.
	\item Register names were adjusted (cost of not using peripheral libraries).
	\item This board does not have external SDRAM.
	\item This microcontroller's maximum SPI speed is 42 MHz.
\end{itemize}

%%%%%%%%%%%%%%%%%%%%%%%%%%%%%%%%%%%%%%%%%%%%%%%%%%%%%%%%%%%%%%%%%%%%%%

\subsubsection{Threaded programming}
\begin{itemize}
	\item Threaded programming was needed because reading the frames from the camera requires very little processing power.
	\begin{itemize}
		\item Just waiting for the STM32 is underusing Raspberry resources.
	\end{itemize}
	\item POSIX threads were chosen for the task.
	\item Using real-time linux scheduling policies.
	\item The skeleton code is done.
	\item Tuning needs to be done when everything is ready.
	\item Only three threads are used at the moment.
	\begin{itemize}
		\item \textbf{Capture:} Reads frames from the camera to memory from the SPI peripheral. Signals the STM32 boards to keep the synchronization.
		\item \textbf{Depth:} Commands the GPU to perform the multiple rendering stages and get the resulting disparity map back to CPU memory.
		\item \textbf{Lane:} Executes the Hough Transform.
	\end{itemize}
\end{itemize}
